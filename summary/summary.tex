\chapter{総括}
本研究では,経験的ペアポテンシャルをでの計算を前提としているMoment法にVASPによる第一原理計算の結果の導入を試みた.
計算は対称性に優れており等方的な格子を持つfcc構造金属であるCu, Ag, Au, Alを対象とした.
Moment法の熱膨張の計算原理は線形結合を前提としている.従来のペアポテンシャルの計算では
原子をfcc構造で配置させそれぞれの原子に対して,x方向の2次,4次微分の総和をとることで線形結合に
対応していた.
それに対してVASPでの導入ではそれぞれの元素のユニットセルを格子の長さを変化させ第一原理計算を行い,結果からフィッティングにより関数を作る.
フィッティングにより得られたポテンシャル関数は3次元での計算結果であり,線形結合の熱膨張に対応させなければいけない.
そこで今回はfcc構造は等方性に優れているという点から3で割ることによって線形結合への対応を試みた.
これにより得られた計算に必要なパラメータ$k$, $\gamma$を用いて熱膨張,自由エネルギーの計算をした.
計算の信頼性を確かめるために,従来のペアポテンシャルを利用したMoment法,MedeA,PhonopyによるPhonon-DOS法と比較をおこなった.

結果をまとめると以下のようになる.
\begin{itemize}
 \item VASPを導入したMoment法は従来のペアポテンシャルのMoment法よりも,全ての計算でMedeA, Phonopy, 実験値に近い値をだした.
 \item Auの熱膨張はMedea, PhonopyではPhonon状態密度に負の値が混じり上手く再現できなかった.しかし,Phononとは違うアプローチで計算をするVASPを導入したMoment法では実験に近い数値を出すことができた.
  \item Alにおける熱膨張は,Phonopy, MedeAではよく再現することができている.しかし,VASPを導入したMoment法では熱膨張が小さいという結果になった.
 \item 内部エネルギーと熱膨張を考慮に入れたPhonopyの自由エネルギーの結果と比較すると,Cu, Agは比較的一致を見せた,Au, Alにおいては熱膨張に差があることもあり異なるカーブを描いた.
\end{itemize}
以上の結果より,VASPを導入したMoment法はある程度信頼できるということがわかった.

今後の課題としては,

\begin{itemize}
 \item 今回はfcc構造の等方性に注目し線形結合に対応するためにポテンシャルを3で割るという手法を試みた.これに対して,もっと良い手法がないか検討する.
 \item 今回の計算にはポテンシャルをフィッティングする際に7次の項まで利用したが,拾えていない成分が残っているかもしれない.そのため,フィッティング精度を高めてさらに高次の項まで取り込んだ計算を行えば違う結果が得られる可能性がある.
  \item 等方的ではないhcp構造での実装はどうするのか,$a$軸,$c$軸方向のポテンシャルは作ることができるが,それらからどのように実装するか考える必要がある.またそれにより負の膨張率の再現することができるか検証が必要である.
  \item $y_0$を近似によって求めているがもっと良い近似法がないか検証する.
\end{itemize}
以上が挙げられる.



