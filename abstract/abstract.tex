\begin{abstract}
材料設計において系の有限温度における自由エネルギーの変化は基本となる物性値である.
第一原理計算は絶対零度での計算のため熱振動の効果を考慮していない.
しかしながら,熱膨張,比熱,電気伝導率などの諸物性は有限温度において振動の影響を受ける. 
そのため,有限温度での物性を計算により求めることは,
振動の効果を取り入れて計算することが必要となる.

第一原理計算ソフトVASP(Vienna ab initio simulation package) では,
擬調和振動子近似に基づいたphonon計算パッケージが開発されており,
Phonon-DOS法を用いて自由エネルギーを見積もることができる.
しかし,Ti 結晶での相変態温度近傍での振る舞いと体積膨張率において実験を再現する結果が得られない.
これは相互作用の高次項である非調和項の影響が大きいためと考えられる.
%これは現実的な結晶の振る舞いを再現するためには無視することができない成分である.

Vu Van Hung らが開発したMoment 法では,
原子間の相互作用エネルギーの高次微分を考慮にいれることによって,
非調和効果を取り入れた有限温度における自由エネルギー,熱膨張などを見積もることができる.

本研究では等方的な格子構造を持つfcc金属であるCu, Ag, Au,Alを計算対象としMoment法へのVASPの導入を試み,熱膨張,自由エネルギーの計算をおこなった.
VASPの支援ソフトであるMedeAとオープンソースのPhonon計算ソフトであるPhonopyを用いて,
Phonon-DOS法による熱膨張,自由エネルギーを求め,それらと比較することで計算結果の信頼性を確かめた.
結果は,ペアポテンシャルを用いた従来のMoment法と比べると熱膨張,自由エネルギーともにMedeA, Phonopyに近い数値を出すことができた.
この手法はまだまだ改善の余地があり,今後の計算精度の向上に期待できる.
 \end{abstract}
