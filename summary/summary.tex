\chapter{総括}
本研究では,経験的ペアポテンシャルをでの計算を前提としているMoment法にVASPによる第一原理計算の結果の導入を試みた.
計算は対称性に優れており等方的な格子を持つfcc構造金属であるCu, Ag, Au, Alを対象とした.
Moment法の熱膨張の計算原理は線形結合を前提としている.従来のペアポテンシャルの計算では
原子をfcc構造で配置させそれぞれの原子に対して,x方向の2次,4次微分の総和をとることで線形結合に
対応していた.
それに対してVASPの計算はfcc構造が得られそこからフィッティングをし関数を作る.
そこから,fcc構造は東方性に優れているため3で割ることによって線形結合への対応を試みた.





\begin{enumerate}
  \item 分子動力学法による粒子の動きをシミュレーションし視覚化することによって,クラスタや凝固などの現象を視認することができ直感的な理解が可能となった.
また,粒子をマウスで操作できるため,自分の好きなようにシミュレーションでき理解の向上に繋がると考えられる.
  \item プログラムのJavaScript化を行いWebブラウザ上で動作が可能になり容易に公開,利用することができる.
そのため,分子動力学法を学習する者が手軽に扱うことができ,学習意欲,効率の向上に繋がると考えられる.
  \item 作成したプログラムをライブラリとして保存しプログラムの解説を行うことで,プログラムの継続的な発展を可能にした.
また,今後シミュレーションプログラムを作成する際に今回のプログラムをライブラリとして利用し効率よく作成する事が可能である.
\end{enumerate}



